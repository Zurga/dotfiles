\cvsection{Onderwijs}
\begin{cventries}
  \cventry
    {UvA}
    {Informatiekunde (BSc)}
    {2014 - now}
    {Amsterdam}
    {Informatiekunde is een goede mengeling tussen bedrijfskunde en informatica. Veel bedrijven hebben moeite om ICT goed aan te laten sluiten bij hun business, dit is waar informatiekundigen het veld betreden. 
    Een ander interessant vakgebied dat aangesproken wordt tijdens de studie is omgaan met informatie en kennis. Dit zowel in het vergaren als representeren. Een deel van de data analyse methodes die bij Kunstmatige Intelligentie aan bod komen worden ook in deze studie gegeven.}
    {}

  \cventry
    {UvA}
    {Algemene Sociale Wetenschappen (BSc)}
    {Amsterdam}
    {2011 - 2012}
    {Bij Algemene Sociale Wetenschappen wordt de nadruk gelegd op interdisciplinariteit tussen de verscheidene sociale wetenschappen.}

  \cventry
    {UvA}
    {Kunstmatige Intelligentie (BSc)}
    {2010 – 2011}
    {Amsterdam}
    {Het menselijk denken om kunnen zetten in een taal die de computer begrijpt is een van de grote uitdagingen binnen Kunstmatige Intelligentie. Daarnaast wordt er ook veel aandacht besteedt aan zelflerende applicaties en het ontwerpen van geavanceerde data analyses.}

  \cventry
    {}
    {VWO N+G Joke Smit Amsterdam (diploma behaald)}
    {2009 - 2010}
    {Amsterdam}
    {}
\end{cventries}
